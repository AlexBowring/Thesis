In this thesis, we have focussed our attention on some of the key issues at the forefront of task-based functional magnetic resonance imaging (fMRI). Our work has been carried out at a pivotal moment in the field's history, during the emergence of population-size neuroimaging studies that are delivering functional data on an unprecedented scale. While fMRI has traditionally been a small data enterprise, where sample sizes of 20 to 30 subjects are common for a task-based study, datasets such as the UK Biobank and Human Connectome Project are now giving researchers the opportunity to analyze fMRI data acquired from tens of thousands of participants. Population neuroimaging projects promise to transform our understanding of brain function, and are already yielding rich results \citep{Miller2016-hd, David_C_Van_Essen2016-bt}. However, the arrival of these datasets have also made this a critical time to reassess the current analysis methods implemented for task-based fMRI, as well as presenting new questions about how to appropriately analyze data of such magnitude.

The challenges posed by the big data revolution have guided all aspects of this effort. With the plurality of tools and techniques being applied to analyze population-size datasets, there has been a growing apprehension within the field about the scale of analytic variability between neuroimaging results. In Chapter \ref{chap:software}, we helped address these concerns in regards to variation between fMRI analysis software. By reanalyzing three publicly available fMRI datasets in the three most popular analysis packages, we ultimately demonstrated the fragility of group-level fMRI results dependant on the software package chosen to analyze the data. Our main contribution here was the measurement of inter-software differences using a variety of quantitative methods; Dice coefficients, Bland-Altman plots and Euler Characteristic curves illuminated disparities between the final group-level statistic maps obtained with each package in terms of the location, magnitude and topology of each software's activation profile. 

Our findings that weak effects may not generalize across software have highlighted the need for greater examination of pipeline-related variation to detect inconsistencies between packages. While here we focussed on the statistic maps obtained at the end of an analysis, a limitation of this approach was that it meant our comparisons reflected the net accumulation of differences across the \textit{entire} pipeline, rather than diagnosing the precise procedures where the greatest variation between software transpired. In future work we plan to carry out further analyses on the three datasets used in this effort, implementing a common preprocessing strategy before utilizing procedures from different packages in the remaining stages of the analysis. This will distinguish whether the largest sources of software-variability are during the preprocessing or statistical modelling of fMRI data, prompting a further assessment of individual procedures to find areas where the three packages can be harmonised. 

Alternatively, given that each analysis package represents a peer-reviewed and valid analytic strategy, another line of work may include the development of techniques to synthesize inconsistent findings. While meta-analysis methods are routinely used in fMRI to aggregate data from separate studies, our setting is unique in the sense that all results are obtained from a \textit{single} source dataset. Therefore, and unlike traditional meta-analyses, any inter-result variation is solely due to methodological differences between analysis software. As part of our future work, we look to expand on current approaches to develop a series of `same-data meta-analysis' techniques, aiming to provide a consensus result across software calibrated to have an image-wise mean and variance similar to the individual packages' group-level statistic maps. These sort of methods may benefit further efforts to combine data from replication studies, such as the \textit{Neuroimaging Analysis Replication and Prediction Study} \citep{Botvinik-Nezer2019-qu}, where a single fMRI dataset has been analyzed by multiple teams worldwide. 