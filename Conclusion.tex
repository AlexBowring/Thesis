In this thesis, we have focussed our attention on some of the key issues at the forefront of task-based functional magnetic resonance imaging (fMRI). Our work has been carried out at a pivotal moment in the field's history, during the emergence of population-size neuroimaging studies that are delivering functional data on an unprecedented scale. While fMRI has traditionally been a `small-data' enterprise, where sample sizes of 20 to 30 subjects are still common for a task-based study, datasets such as the UK Biobank and Human Connectome Project are now giving researchers the opportunity to analyze fMRI data acquired from tens of thousands of participants. These population-level neuroimaging projects promise to transform our understanding of brain function, and are already yielding rich results \citep{Miller2016-hd, David_C_Van_Essen2016-bt}. However, the arrival of these datasets have also made this a critical time to re-assess the current analysis methods implemented for task-based fMRI, as well as presenting new challenges about how to appropriately analyze data of such magnitude. It is these specific issues that have ultimately guided all of our work in this effort. 

In the first part of this thesis we addressed concerns about the software-reproducibility of task-fMRI results across the three main fMRI analysis packages. 