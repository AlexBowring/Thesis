In this chapter, we provide the context that forms the basis of our research. We begin by presenting a broad overview of the study of brain function, before narrowing down to the specific field of task-based functional Magnetic Resonance Imaging (t-fMRI) that will be the main focus of study in this thesis. Here, we describe each of the preprocessing and modelling components of a typical t-fMRI analysis pipeline. Finally, we give an in-depth discussion of the state-of-the-art procedures used for subject- and group-level t-fMRI inference that are of particular relevance to the remaining chapters of this work. 

\pagebreak

\section{The Study of Brain Function}

\begin{figure}[htbp]
\centering
	\includegraphics[height=3in]{1_Grays_Anatomy.png}	
\caption*{\textbf{Figure 1:} An illustration showing the arteries at the base of the human brain from (Gray, 1918).}
\end{figure}

The human brain, the central organ of the human nervous system, has been described as one of the most complex structures in the known universe. Made up of approximately 86 billion neurons \citep{Azevedo2009-qj}, where neuronal interaction occurs continuously via trillions of synaptic networks to form intricate and dynamic neural networks, the myriad of processes taking place inside the brain at any given time make the study of brain function an intimidating challenge. Nonetheless, our understanding of this organ has come along way from our ancient Egyptian ancestors, who believed that the heart was at the source of human intelligence, and for whom the practice of drilling a hole into the skull was regarded as a solution to cure a headache \citep{Adelman1987-hs, Mohamed2014-gl}. 

Remarkably, much of this progress has come in the last century alone. A number of key developments within this time-frame include: Confirmation of the neuron doctrine, the concept that the nervous system is a collection of discrete individual cells, postulated by Santiago Ramon y Cajal at the end of the 19th century and demonstrated in the 1950s thanks to the development of electron microscopy \citep{Lopez-Munoz2006-zk}; the first evidence of neuroplasticity, the ability for the brain's structure to change during an individual's lifetime \citep{Diamond1964-cu, Bennett1964-vx}; and the emergence of neuroimaging techniques such as electroencephalography (EEG), positron emission tomography (PET), and MRI. The toils of this scientific endeavour are now translating into concrete advancements influencing a wide variety of aspects concerned with population health. Neuroscience research is beginning to find applications in the clinical setting to advance our understanding of neurodevelopmental and neurodegenerative disorders and generate novel therapies to treat and prevent such diseases. Brain imaging has been used to localize the source of neurological impairment for diseases such as epilepsy \citep{Stacey2008-qx}, and neuroengineering techniques based on our capability to stimulate neural circuits are implemented to treat Parkinson's disease \citep{Kalia2013-bv} and dystonia \citep{Fox2015-ds}. Structural- and functional-MRI are being explored to determine biomarkers for diagnosis of Alzheimer's disease \textit{prior} to symptom onset \citep{Sperling2014-sy, McEvoy2009-zx}, alongside providing information about the role of different brain regions in human behaviour that can contribute to an improved prognosis and patient response to therapy \citep{Matthews2006-sl}.  

Modern neuroscience can be dissected into many major branches, each subfield taking a specific slant to studying the nervous system. It is therefore perhaps unsurprising that in isolation, the phrase `the study of brain function' is rather vague. Brain function can manifest itself in ways that can be observed using a variety of different measurements, whether that be with a molecular, chemical, structural, or functional approach \citep{Hargreaves2012-dz}. Different modalities of MRI are employed to evaluate specific properties that ultimately characterize whichever approach is taken. For instance, looking at brain function from an anatomical perspective, voxel-based morphometry (VBM) could be used to measure differences in local concentrations of brain tissue, to assess, for example, changes in grey matter volume \citep{Mechelli2005-dn}. Additionally, one could apply diffusion tensor imaging (DTI) to instead map white matter tractography in the brain \citep{Alexander2007-ut, Soares2013-mh}. From a functional outlook, resting state fMRI (rs-fMRI) determines that spatially remote brain areas are functionally connected when each region's BOLD response is temporally correlated in the absence of an explicit task \citep{Lee2013-kn}. On the other hand, task-based fMRI (t-fMRI) measures spatio-temporal changes in the BOLD signal between task-stimulated and control states to find brain regions that are activated in the presence of a stimulus \citep{Glover2011-at}. 

Each imaging method and modality does not live inside a vacuum, and recent work within the field has provided further insight of the interdependence between different approaches to examining brain function. One example of this is in the study of resting state networks, which explores how distinct sets of brain regions can reveal temporally correlated activation patterns when the brain is at rest. While resting state networks have been most widely investigated using rs-fMRI techniques \citep[e.g.][]{Smith2009-dm, Lee2012-di, Moussa2012-bl}, more recently, the same correlation patterns have been independently detected using EEG and MEG \citep{Brookes2011-cj, Fomina2015-ha}. This work not only demonstrates how utilization of numerous tools can further our understanding of resting state mechanisms, but also suggests a direct relationship between the electro-physiological signals recorded with MEG and the BOLD fluctuations associated to fMRI. Similarly, other recent efforts have shown that the functional response to a cognitive task measured with t-fMRI may be able to be predicted by connectivity features from the same individual's brain at rest \citep{Parker_Jones2017-ld, Tavor2016-pd}. This research signals towards an innate functional signature that defines our behaviour, while also providing potential clinical solutions to obtain t-fMRI data from patients who are unable to perform the specific task of interest.  

In the context of this thesis, we will study brain function from a functional perspective, primarily focussed on task-based fMRI. 

\section{Blood Oxygenation Level Dependant (BOLD) Functional Magnetic Resonance Imagery (fMRI)}

Whereas structural MRI is concerned with the anatomy of the brain, functional MRI (fMRI) measures dynamic changes in blood flow in order to ultimately make inference on neuronal activation. This is possible due to the intrinsic relationship between local neuronal activity and subsequent changes in cerebral blood flow (CBF), a biological phenomenon known as neurovascular coupling. An increased supply of oxygen is carried by haemoglobin in red blood cells to provide energy to active neurons, and it is the magnetic properties of the haemoglobin that MRI takes advantage of. Specifically, as deoxygenated haemoglobin is more magnetic (paramagnetic) than oxygenated haemoglobin, MRI uses haemoglobin as an endogenous contrast agent from which to source the signal. Neurovascular coupling induces inhomogeneities in the local magnetic field, primarily due to a concentration of deoxygenated haemoglobin in activated brain regions, that lead to a detectable change in the MR signal.

The complete chain of events linking neuronal activity to a change in MRI signal is referred to as the Blood Oxygenation Level Dependant (BOLD) effect, and this type of imaging is known as BOLD fMRI.  Proof of concept of the BOLD effect was first provided in \citet*{Ogawa1990-it}, and the first use of BOLD fMRI for human brain mapping was carried out in 1992 \citep{Bandettini1992-jt, Kwong1992-uq, Ogawa1992-af}, leading to a large uptake of the method that has continued to this day. Alternative approaches to functional imaging exist, the most popular of which is functional Arterial Spin Labelling (fASL), that uses magnetically labelled arterial blood water to quantify changes in CSF. While fASL can offer some advantages over fMRI, and changes in CSF measured with the technique are more closely tied to neuronal activation than the BOLD signal, fASL suffers from a much lower signal-to-noise ratio that consequently has made fMRI the preferred imaging modality of choice. 

\subsection{Physiology of the BOLD response}

\section{Task-based functional Magnetic Resonance Imagery (t-fMRI)}

\subsection{Overview}

\subsection{Pre-processing}

\section{Statistical Analysis: Subject-level}

\subsection{Parametric Methods}

\subsection{Nonparametric Methods}

\section{Statistical Analysis: Group-level}

\subsection{Parametric Methods}

\subsection{Nonparametric Methods}

\section{Reproducibility of fMRI Results}

