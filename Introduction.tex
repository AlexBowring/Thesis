Since its inception at the end of the twentieth century, functional Magnetic Resonance Imaging (fMRI) has experienced a meteoric rise to become the primary tool for human brain mapping. While many forms of the technique exist, introduction of the particular method based on the Blood Oxygenization Level Dependant (BOLD) effect has ultimately proven to be the catalyst in elevating fMRI to such stature within the neuroimaging community. Taking advantage of the magnetic properties of oxygen-rich red blood cells, BOLD fMRI measures changes in blood oxygenization alongside cerebral blood flow and volume as a proxy to identify brain areas where elevated neuronal activity has occurred in response to a stimulus. While the relationship between the BOLD effect and neuronal activity is complex and remains controversial, it is the unique attributes of BOLD fMRI -- in particular, its capacity for non-invasive recording of signals across the entire brain at a high spatial resolution -- that set the technique apart from other scanning methods.

However, BOLD fMRI is also a \textit{noisy} process. The MR signal researchers set out to measure during a scanning session is corrupted by artefacts from both the imaging hardware and the physiology of the participant. Examples of scanner noise include inhomogeneities of the magnetic field that can cause spatial distortion or blurring in the MR image, and scanner drift characterized by temporal degradation of the signal. Physiological noise induced by subject motion, respiration, and heartbeat exacerbate the problem. 

Because of the low signal-to-noise, researchers must apply a series of statistical techniques to find meaning in the data. This usually entails carrying out a number of preprocessing, modelling and analysis steps which together constitute the fMRI processing pipeline. The fundamental objectives of preprocessing are to standardize brain locations across participants, to apply methods ensuring that the data conform to statistical assumptions required for analysis, and to reduce the influence of the aforementioned noise artefacts present in the data. This is achieved by conducting a number of steps, including slice-timing correction, motion correction, normalization, registration of the functional data to an anatomical template, and spatial smoothing. 

For task-based fMRI, a mass-univariate approach is utilized to model the data. During the scanning session, functional data are acquired in the form of voxels -- cubic intensity units that partition the brain comparable to the way in which pixels partition a computer screen. Each voxel's time-series is considered independently within the general linear model framework as a combination of signal components. To evaluate the effect of an experimental task condition relative to a baseline condition, hypothesis testing is performed at each voxel to compute a statistical parametric map of $t$-statistic values. Here, the behaviour of the signal under the null hypothesis of no activation is estimated using either a parametric approach, appealing to the body of mathematics known as Random Field Theory, or a non-parametric approach, where permutation-methods are applied to estimate the null-distribution directly from the data. Finally, the statistical parametric map is thresholded to localize brain function. 

While we have provided a brief overview of the fMRI analysis pipeline, it is notable that there is not a general consensus as to how each particular analysis step should be carried out. Consequently, researchers have the freedom to make many choices during an analysis, such as how much smoothing is applied to the data, or how the heamodynamic response of blood flow to active neuronal tissues is modelled. However, this `methodological plurality' comes with a drawback. While conceptually similar, two different analysis pipelines applied on the same dataset may not produce the same results. In perhaps the most comprehensive study on this topic (Carp, 2012, Plurality...), a single publicly available fMRI dataset was analyzed using over 6,000 unique analysis pipelines, generating 34,560 unique thresholded activation images. These results displayed a substantial degree of flexibility in both the sizes and locations of significant clusters. Further differences in methods, implementations across software packages, operating systems and even software versions can contribute to variability in the final scientific outcomes (add refs). 

The problem is, with so many statistically valid methodological strategies available, if you try them all you are likely to find \textit{something}. The high analytic flexibility associated to fMRI has therefore been pinpointed as a key risk factor for inflated false positive results (Ioannidis, 2005, why most research findings are false). This may also be compounded by publication bias, for which there has been evidence to suggest that studies finding significant outcomes are disproportionately represented within the fMRI literature. 

%The variability of such methodological decisions is high between %published studies in the fMRI literature (Carp, 2012 (secret lives)).