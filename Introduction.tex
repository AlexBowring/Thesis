Since its inception at the end of the twentieth century, functional Magnetic Resonance Imaging (fMRI) has experienced a meteoric rise to become the primary tool for human brain mapping. While many forms of the technique exist, introduction of the particular method based on the Blood Oxygenization Level Dependant (BOLD) effect has ultimately proven to be the catalyst in elevating fMRI to such stature within the neuroimaging community. Taking advantage of the magnetic properties of oxygen-rich red blood cells, BOLD fMRI measures changes in blood oxygenization alongside cerebral blood flow and volume as a proxy to identify brain areas where elevated neuronal activity has occurred in response to a stimulus. While the relationship between the BOLD effect and neuronal activity is complex and remains controversial, the method's    