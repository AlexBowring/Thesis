Since its inception at the end of the twentieth century, functional Magnetic Resonance Imaging (fMRI) has experienced a meteoric rise to become the primary tool for human brain mapping. While many forms of the technique exist, introduction of the particular method based on the Blood Oxygenization Level Dependant (BOLD) effect has ultimately proven to be the catalyst in elevating fMRI to such stature within the neuroimaging community. Taking advantage of the magnetic properties of oxygen-rich red blood cells, BOLD fMRI measures changes in blood oxygenization alongside cerebral blood flow and volume as a proxy to identify brain areas where elevated neuronal activity has occurred in response to a stimulus. While the relationship between the BOLD effect and neuronal activity is complex and remains controversial, it is the unique attributes of BOLD fMRI -- in particular, its capacity for non-invasive recording of signals across the entire brain at a high spatial resolution -- that set the technique apart from other scanning methods.

Unfortunately, BOLD fMRI is also a \textit{noisy} process. The MR signal researchers set out to measure during a scanning session is corrupted by artefacts from both the imaging hardware and the physiology of the participant. Examples of scanner noise include inhomogeneities of the magnetic field that can cause spatial distortion or blurring in the MR image, and scanner drift characterized by temporal degradation of the signal. Physiological noise induced by subject motion, respiration, and heartbeat exacerbate the problem. 

In essence, fMRI has a low signal-to-noise ratio, and researchers are therefore dependant on statistical techniques to find meaning within the data. This usually entails carrying out a number of preprocessing, modelling and analysis steps which together constitute the fMRI processing pipeline. The fundamental objectives of preprocessing are to standardize brain locations across participants, to apply methods ensuring that the data conform to statistical assumptions required for analysis, and to reduce the influence of the aforementioned noise artefacts present in the data. This is achieved by conducting a number of steps, including slice-timing correction, motion correction, normalization, registration of the functional data to an anatomical template, and spatial smoothing. 

For task-based fMRI, data are modelled subsequent to preprocessing by utilizing a mass-univariate approach. During the scanning session, functional data are acquired in the form of voxels -- cubic intensity units that partition the brain analogous to the way in which pixels partition a computer screen. To analyse the data, each voxel's time-series is considered independently within the general linear model framework as a linear combination of signal components. Following this, null-hypothesis testing is conducted at each voxel to compute a $t$-statistic contrasting a specified experimental task condition relative to a baseline condition. Both parametric and non-parametric methods can be used to 